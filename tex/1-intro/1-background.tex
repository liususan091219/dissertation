%!TEX root =../../thesis-ex.tex



\section{Background}
\label{ch1:sec1:background}

\subsection{Studies in Decision Making}



\subsection{Traditional Decision Support Systems}

\subsection{Data-Driven Information Systems}

Search engine and recommender system are the two major types of information systems that users frequently interact with for making decisions. Such work mostly inspires the methodology of this thesis on how to develop data-driven models to support users for making decisions. 

\textbf{Learning from User Click Logs}. User click through logs are regarded as partial relevance feedback, therefore they can be used to train machine learning and neural network models which effectively improve the performance of actual ranking results compared with non-learning approaches. Such models are further improved, because the clicks themselves may not directly reflect user satisfaction. For example, if the user visits a link for less than a few seconds, it is less likely that she has found the relevant information from that link. As a result, people have proposed to focus on clicks with longer dwell time (e.g., exceeding 30 seconds) which are more likely satisfied clicks~\cite{kim2014modeling}. On the other hand, even if the user skips a result, it does not necessarily mean it is non-relevant. It may be because the relevant information need already appeared on top. To further improve the model, \cite{Craswell:2008:ECC:1341531.1341545} proposed a cascade model which captures the probability for skipping the top results. 

\textbf{Leveraging Other Data Resources}. In information systems, meta data/additional user generated data can often help with search/recommender system optimization, especially when no query is available or the query is too simple. For example, user's biographic features, such as name, gender, can help with improving the performance of personalized recommender system. User review data, provides crowd sourced opinions that helps with identifying high quality results. For instance, leverage topic modeling on products review data to improve the matching probability between query and products, as products are structured data which often lacks opinionated descriptions as in the queries, e.g., \emph{quiet fan}~\cite{duan2013probabilistic}. User review data can also support aspect based search to cater users' fine grained needs, e.g., when searching for hotels, some users consider location an important factor, while others care more about price. 

% \textbf{Formal Models of Users' Information Seeking Behaviors}. Researchers have built different models to capture users' information seeking behavior. An early model proposed by Pirolli and Card compares users' information seeking process with predators (user) hunting for food, where predators consume energies (searching and browsing) in hoping to finally reach the food (information need is satisfied). In web search and browsing, users behaviors consists of frequent switches between searching, scanning highlights, visiting websites, consumption of the contents, leaving sites, etc., until the user's information goal is reached or the user gives up. During these processes, it is the \emph{information scent} that supports users to go deeper, for example, relevant texts and highlighted words on search engine result page give users the promise that they may eventually get the ``food'' by visiting a website. In other words, unless user finds the scent stronger and stronger along the path (i.e., improved quality), she will give up the process. Such theory had turned into concrete guidelines to help websites improve the designs. It also inspired Google to re-rank the search results~\cite{Nielson2003Forage}. Following the information scent work, researchers at PARC develop an automated usability testing tool called Bloodhound~\cite{chi2003bloodhound}, which uses a probabilistic model to measure the easiness for user to reaching the desired destination. User studies show that the proposed measure agrees with users' real responses.

% Information foraging theory was adopted by more recent work in formal models of users' information seeking. SNIF-ACT~\cite{fu2007snif} test user actions in a world wide web setting. \cite{chi2010method} further developed methods for highlighting texts. Multiple systems was proposed to estimate the semantic relatedness to simulate information scent~\cite{budiu2007modeling}, e.g., latent semantic analysis by (LSA~\cite{landauer1997solution}). Information scent was also applied to the scenario of software development~\cite{lawrance2008using}. The theory inspired a list of work on using simulation to replace human usability testing to reduce the cost of testing~\cite{chi2003bloodhound,carterette2015dynamic}. For example, \cite{carterette2015dynamic} simulates a complete set of behaviors, including clicks, dwell time, abandonment, and query through probabilistic models. \cite{luo2014win} presents a similar idea by modeling the transition of search engine states using partially observable Markov decision processes (POMDP). 

% More recently, researchers have developed other formal models for the users' interactive searching process~\cite{conf/sigir/ZhangZ15,zhang2016sequential}. In these models, the search process is modeled as a collaborative game playing process between user and the search engine. The two parties take turns to play a card, and the goal is for them to achieve the best reward with the least cost. Such models are general enough to not only model the web search process of ten blue links in the search engine result page, but they can also be extended to the case of mobile search when SERP is not available, and when the interface consists of more components to support navigation~\cite{zhang2017information}. When playing the card, the search engine can update the interface in real time by actively choosing between showing the content and showing the navigation component, which is beneficial when querying plays a less important role, e.g., news browsing.  

\subsection{Mobile User Interactions}

The last decade witnessed a revolutionary increase in mobile market, the penetration rate almost doubled within the 10 years between 2007-2017, making more than 66.53\% of the world population own a mobile device in 2019~\cite{phone_penetration}. Statistics show that people spend approximately 4 hours a day on their phone. Because we have developed such a close relationship with our phone, researchers have been working on studying how users interact with mobile phones and how we can optimize such interactions. For example, a significant proportion of the SIGCHI proceedings each year are related to user interactions with mobile devices (Figure~\ref{ch1:fig1:sigchi_stats}). 

\begin{figure}[h]
\centering
\subfloat{
\begin{tikzpicture} [scale=0.9]
\begin{groupplot}[group style={group size= 1 by 2},height=5cm,width=10cm]%[ybar stacked,xtick=\empty,]%ytick=\empty]
\nextgroupplot[ybar,symbolic x coords={2006, 2007, 2008, 2009, 2010, 2011, 2012, 2013, 2014, 2015, 2016, 2017, 2018, 2019},legend style={at={(0.5,-0.2)},anchor=north, ymin=0, ymax=120,legend columns=2},  ymajorgrids = true,bar width = 4.5,xtick=data, x tick label style={rotate=45,anchor=east}]%ytick=\empty]
\addplot[fill=black,draw=black] 
coordinates {(2006, 24.0) (2007, 23.0) (2008, 24.0) (2009, 31.0) (2010, 37.0) (2011, 50.0) (2012, 52.0) (2013, 49.0) (2014, 78.0) (2015, 75.0) (2016, 68.0) (2017, 73.0) (2018, 100.0) (2019, 81.0)};
\end{groupplot}
\end{tikzpicture}
}

\caption{Number of SIGCHI proceedings over the years with title containing the word \emph{mobile} or \emph{phone}\label{ch1:fig1:sigchi_stats}}
\vspace{-0.2in}
\end{figure}

Mobile devices differ from laptop/desktop computers in many aspects, including both the operating system and the user interface. For both aspects, researchers have studied the impact of such difference on users' interactive behaviors. 

\textbf{Users' Mobile Search Behaviors}. Compared with desktops/laptops, mobile devices have much smaller screens and it is easier to make mistakes during typing. As a result, users often show very different search behaviors. 

Previous work conduct large-scale empirical studies on Google~\cite{kamvar2006large} and Yahoo! search logs~\cite{yi2008deciphering}. The former study found that users' exploratory behaviors in mobile searches are largely lowered~\cite{kamvar2006large}. Previous work has not found a large difference in query lengths on mobile devices and computers, but compared with on mobile devices, users tend to reformulate more queries in the same session on computers~\cite{kamvar2006large}. Such results are consistent with the fact that mobile typing is more difficult than typing on computers. Another work used eye-tracker to record the difference between the eye movement behaviors on mobile devices and computers~\cite{kim2015eye}. They find that users exhibit slower eye movements on mobile devices than on computers. However, they experience more difficulty extracting information on mobile devices, and users are more likely to focus on top-ranked results on mobile devices. Users' mobile search behaviors also verifies the theory of information scent, where researchers found that mobile searchers need an increased amount of relevant search results, while desktop searchers are more accurate when each page contains an equal number of relevant search results~\cite{ong2017using}. Another behavior difference lies in \emph{good abandonment}, which means the user already finds an answer in the search engine result page, therefore the information need has been satisfied before any clicks. Researchers found such behaviors more frequent on mobile devices~\cite{williams2016detecting}. As a result, user satisfaction is not determined solely by clicks, therefore they propose to use gesture features to estimate users' satisfaction.  

The difference between mobile and desktop/laptop computers has also inspired research of actionable results. \emph{Summarization}. With smaller screens, mobile information systems no longer display the complete long text description as on desktop, e.g., the title of e-Commerce products can be summarized to better fit in the smaller screen on mobile devices~\cite{sun2018multi}. \emph{Search result diversification}. Since mobile users lack exploration and query reformulation, the search engines provide remedies. Notably, the search results on mobile devices are more diversified than on computers to encourage exploration. \emph{Enhanced query auto completion}. As researchers observe the difficulty for typing, they propose a term-by-term strategy for auto completing queries, different from the standard strategy which suggest the whole query at the same time~\cite{vargas2016term}. \emph{Context-aware Results}. The location of user provides information that could be leverage to improve the results in multiple aspects~\cite{lin2017location}. \emph{Slow search}. As mobile search tends to be slower than desktop search due to the network condition or other factors, researchers proposed to include higher quality results trading off the delay time.

\textbf{User Behaviors towards Mobile Applications}. Besides the search engine, a major part of mobile user interaction is with mobile applications. Mobile operating systems are dominated by iOS and Android, where Android has more than 76\% of the market share. As of 2019, Android has released its 10th version (Android Q). 

A large part of the user behavior studies on mobile applications focus on their behaviors towards security and privacy operations. Because such operations directly affects users' own security, users often need to spend time making decisions, therefore users' security response can lead to direct implication which helps system and application developers improve the security of the system. 

In 2011, researchers found that users were confused by Android permission requests, having trouble making decisions on whether to grant permissions for an application~\cite{conf/soups/FeltHEHCW12}. This behavior is due to the design of Android system, where applications must ask for permissions from the user before they can get access to the corresponding resources. The user, as a result, must decide whether to grant permissions to an app. If a permission is unrelated to or looks like it is unrelated to an app's main functionality, it is hard for the non-expert user to determine whether it is a case of \emph{over-privilege} or not (i.e., the application requests more permissions than it needs to). In particular, \cite{conf/soups/FeltHEHCW12} found that more than 1/3 of the apps contain at least 1 application that could not be understood by the user. 

Over the years, Google has released multiple new versions of Android permission systems. After Android 6.0 (Marshmallow), the permission model was replaced by a new model where each permission is requested one at a time and during runtime. This design makes it easier for application developers to embed the permission request in context, e.g., requesting location with the background showing a maps makes it easy to understand the motivation. The new system also allows users to turn off a permission after previously granting it. However, researchers still found users' difficulty in understanding permissions~\cite{}, especially with background usages~\cite{background}. 
